% Appendix 1: Personal Projects Developed

\ccvattachment{1}{Anexo I}{\textbf{Projectos Pessoais Desenvolvidos}}

\ccvitem{\textbf{Notification Agenda}\\ Aplicação Android}{\href{http://play.google.com/store/apps/details?id=net.ricardoamaral.apps.notificationagenda}{http://play.google.com/\ldots/details?id=net.ricardoamaral.apps.notificationagenda}}
\ccvitem{Tecnologias Usadas}{Android SDK}
\ccvitem{Sumário do Projecto}{Aplicação simples que transforma pequenas notas em agradáveis notificações na barra de estado com elegantes ícones e de alta qualidade.}

\fancyfoot{}

\tabularnewline

\ccvitem{\textbf{Easy Settings}\\ Biblioteca .NET}{}
\ccvitem{Tecnologias Usadas}{C\# / XML}
\ccvitem{Sumário do Projecto}{Biblioteca .NET fácil e simples de usar para gerir preferências de utilizador de aplicações, guardando-as num ficheiro XML de fácil leitura.}

\tabularnewline

\ccvitem{\textbf{Outros Projectos}}{Para além dos projectos mais relevantes destacados acima, existem outros que podem ser consultados no meu perfil do \href{http://github.com/rfgamaral}{GitHub} e/ou no meu site pessoal; conforme estes tenham sido desenvolvidos num âmbito de código aberto ou não.}

% Appendix 2: University Projects Developed

\ccvattachment{2}{Anexo II}{\textbf{Projectos Universitários Desenvolvidos}}

\ccvitem{\textbf{Sistema de Recomendação de Viagens}\\ Aplicação Windows}{Laboratórios de Informática IV\newline Orlando Belo}
\ccvitem{Tecnologias Usadas}{C\# / UML / SQL Server / ASP.NET / Razor / MVC}
\ccvitem{Sumário do Projecto}{Este projecto consistiu na concepção, análise e especificação de requisitos, desenvolvimento, documentação e manutenção de um sistema de \textit{software}. A ideia apresentada para a realização deste projecto no âmbito do `Sistema de Recomendação de Viagens' proposto foi ``voos de baixo custo na Europa''.}

\tabularnewline

\ccvitem{\textbf{UDP.Friendly}\\ Aplicação Windows}{Comunicações por Computador\newline António Costa}
\ccvitem{Tecnologias Usadas}{C\# / UDP}
\ccvitem{Sumário do Projecto}{O objectivo principal deste projecto consistiu no desenvolvimento de uma camada protocolar sobre o UDP que permitisse regular o débito de dados a transmitir em função dos níveis de carga na rede. Como prova de conceito foram desenvolvidas duas aplicações --- segundo o modelo cliente-servidor --- permitindo a transferência de dados.}

\tabularnewline

\ccvitem{\textbf{Treasure: Planet XPTO}\\ Jogo Windows}{Computação Gráfica\newline António Ramires Fernandes}
\ccvitem{Tecnologias Usadas}{C \& C++ / OpenGL \& GLUT}
\ccvitem{Sumário do Projecto}{Este projecto teve como intuito o desenvolvimento de um jogo ao estilo de um \textit{Role-Playing Game} em primeira e terceira pessoa. O objectivo passou por empregar os conhecimentos adquiridos ao longo do semestre, envolvendo conceitos como sistemas de coordenadas, iluminação, texturas, \textit{Display Lists}, \textit{View Frustum Culling}, VBOs, etc\ldots}

\tabularnewline

\ccvitem{\textbf{Turtly Turtle Movie Database}\\ Aplicação Windows e Web}{Bases de Dados\newline José Manuel Machado}
\ccvitem{Tecnologias Usadas}{C\# / PHP / HTML / CSS / JavaScript \& jQuery / Oracle \& PL/SQL}
\ccvitem{Sumário do Projecto}{Este projecto teve como propósito a aquisição de conhecimentos do SGBD Oracle, tendo como tema a indústria do cinema, mais precisamente os filmes. Foram desenvolvidas duas aplicações, uma para a Web como \textit{front end} para consulta da informação e uma Desktop para administração e manutenção da base de dados.}

\tabularnewline

\ccvitem{\textbf{Algoritmo Simplex}\\ Aplicação Web}{Modelos Determinísticos de Investigação Operacional\newline José António Oliveira}
\ccvitem{Tecnologias Usadas}{PHP / HTML / CSS / JavaScript \& jQuery}
\ccvitem{Sumário do Projecto}{Este projecto consistiu em desenvolver uma aplicação capaz de resolver problemas de programação linear de maximização e minimização. Estes problemas foram resolvidos com recurso a uma implementação do algoritmo Simplex Primal ou Dual.}

\tabularnewline

\ccvitem{\textbf{Só Amigos / Sempre Ligados}\\ Aplicação Linux e Windows}{Laboratórios de Informática III\newline António Nestor Ribeiro \& Luís Paulo Santos}
\ccvitem{Tecnologias Usadas}{C / Java / Swing / MVC}
\ccvitem{Sumário do Projecto}{Associados à mesma Unidade Curricular, foram desenvolvidos dois projectos bastante semelhantes. Ambos consistiram no desenvolvimento ou utilização de estruturas de dados --- recorrendo a diferentes linguagens --- tendo os mesmos como base a implementação de uma versão simplificada da rede social LinkedIn.}

\tabularnewline

\ccvitem{\textbf{AEROGEST}\\ Aplicação Windows}{Programação Orientada aos Objectos\newline F. Mário Martins}
\ccvitem{Tecnologias Usadas}{Java}
\ccvitem{Sumário do Projecto}{Este projecto teve como objectivo aplicar todos os conhecimentos adquiridos ao longo do semestre sobre o paradigma de sistemas de \textit{software} orientado a objectos. O projecto consistiu no desenvolvimento de um ``sistema de gestão de voos''.}

\tabularnewline

\ccvitem{\textbf{Observação}}{Os projectos acima não representam todos os projectos desenvolvidos durante a licenciatura, sendo apenas listados aqueles considerados mais importantes ou relevantes.}