%==================================================================================================
% Workaround for XeTeX compilation with the 'fontspec' and 'europecv' packages
%==================================================================================================

% Pretend that the 'inputenc' package was loaded with the 'utf8x' option

\makeatletter
    \@namedef{ver@inputenc.sty}{}
    \@namedef{opt@inputenc.sty}{utf8x}
\makeatother

\expandafter\let\csname ver@inputenc.sty \endcsname\relax

% Neutralize the '\inputencoding' command used by the 'europecv' package

\providecommand{\inputencoding}[1]{}


%==================================================================================================
% LaTeX markup to typeset a Portuguese Curriculm Vitae using the 'europecv' package
%==================================================================================================

\documentclass[a4paper,portuguese,totpages]{europecv}

\usepackage[top=1.5cm, bottom=1.5cm, left=1.5cm, right=1.5cm]{geometry}
\usepackage[portuguese]{babel}
\usepackage[T1]{fontenc}
\usepackage{fontspec}
\usepackage{enumitem}
\usepackage{graphicx}
\usepackage{hyperref}
\usepackage{xcolor}


%--------------------------------------------------------------------------------------------------
% General configuration for the loaded packages
%--------------------------------------------------------------------------------------------------

\defaultfontfeatures{Ligatures=TeX}
\setmainfont{Calibri}

\setlist{
    leftmargin=*,
    nolistsep,
    itemsep=2pt
}

\definecolor{RoyalBlue}{RGB}{65, 105, 225}

\hypersetup{
    bookmarksopen=false,
    colorlinks=true,
    urlcolor=RoyalBlue,
    pdfpagemode=UseNone,
    pdfstartview={FitH},
    pdfdisplaydoctitle=true,
    pdfinfo={
        Title={Curriculm Vitae de Ricardo Amaral},
        Author={Ricardo Amaral},
        Subject={Curriculum Vitae tipografado em LaTeX com base no modelo Europass}
    },
    pdfcreator={TeXstudio (XeTeX)}
}

% Reduce the amount of lookahead to avoid clearing the footer on wrong pages

\setcounter{LTchunksize}{1}

%--------------------------------------------------------------------------------------------------
% Allow changes to internal LaTeX macros outside packages/classes
%--------------------------------------------------------------------------------------------------

\makeatletter

    % Improve document title style

    \renewcommand*\ecvtitle{\ecv@utf{\Large\textbf{Curriculum Vitae}\\[5pt]\Large\textbf{Europass}}}

    % Improve 'europecv' language table and string styles

    \def\ecv@mothertonguekey{\ecv@utf{\normalsize Língua Materna}}
    \def\ecv@assesskey{\ecv@utf{\normalsize Auto-Avaliação}}
    \def\ecv@levelkey{\ecv@utf{\normalsize Nível Europeu}}
    \def\ecv@listenkey{\vspace{2pt}\ecv@utf{\scriptsize Compreensão Oral}}
    \def\ecv@readkey{\vspace{2pt}\ecv@utf{\scriptsize Leitura}}
    \def\ecv@interactkey{\vspace{2pt}\ecv@utf{\scriptsize Interacção Oral}}
    \def\ecv@productkey{\vspace{2pt}\ecv@utf{\scriptsize Produção Oral}}
    \def\ecv@cefbasickey{\ecv@utf{\scriptsize Utilizador Elementar}}
    \def\ecv@cefindepkey{\ecv@utf{{\scriptsize Utilizador Independente}}}
    \def\ecv@cefprofkey{\ecv@utf{\scriptsize Utilizador Experiente}}
    \def\ecv@langfooterkey{\ecv@utf{\href{http://europass.cedefop.europa.eu/pt/resources/european-language-levels-cefr}{Nível do Quadro Europeu Comum de Referência (CECR)}}}

    \renewcommand*\ecvCEF[2]{
        \begin{tabular}{@{}>{\footnotesize}m{.2\ecv@langparwidth}@{\hspace{1mm}}>{\footnotesize\centering}m{.74\ecv@langparwidth}@{}}
            #1 & #2\tabularnewline
        \end{tabular}
    }

    % Change lists bullet style to a black dot

    \renewcommand{\labelitemi}{$\bullet$}

    % Add a blue underline to URL links

    \Hy@AtBeginDocument{
        \def\@urlbordercolor{0.06 0.3 0.55}
        \def\@pdfborderstyle{/S/U/W 1}
    }

    % Disable the total pages number link to the last page

    \def\ecv@totpages{~/~\ref*{TotPages}}

\makeatother

%--------------------------------------------------------------------------------------------------
% New commands to simplify the CV content markup
%--------------------------------------------------------------------------------------------------

\newcommand{\ccvsection}[2]{\pdfbookmark{#2}{#1}\ecvsection{#2}}
\newcommand{\ccvitem}[2]{\ecvitem[5pt]{#1}{#2}}

\newcommand{\ccvattachment}[4][10pt]{\pdfbookmark{#3}{attachment#2}{\large\par\textbf{#3}}\vspace{#1} & {\par#4}\tabularnewline\nopagebreak}

\newcommand{\weblink}[1]{\href{#1}{#1}}
\newcommand{\maillink}[1]{\href{mailto:#1}{#1}}

%--------------------------------------------------------------------------------------------------
% The beginning of the actual CV document
%--------------------------------------------------------------------------------------------------

\begin{document}

    % Generic Europass CV information

    \ecvfootername{Ricardo Amaral}
    \ecvfootnote{Para se ligar a mim no LinkedIn, por favor visite:\newline
        \weblink{http://www.linkedin.com/in/rfgamaral}}

    % Decrease the total number of pages by the amount of appendix pages

    \addtocounter{TotPages}{-2}

    % The beginning of the Europass CV content

    \begin{europecv}

        % Top left profile photo

        & \ecvspace{-2.2cm} \raggedleft
        \includegraphics[width=3.5cm]{graphics/profile-photo.jpg}
        \ecvspace{-2.2cm} \tabularnewline

        % Personal Information

        \ccvsection{personal}{Informação Pessoal}

        \ccvitem{Nomes / Apelidos}{\large\textbf{Ricardo Filipe Gomes Amaral}}
        \InputIfFileExists{includes/sensible-pt}{}{
            \ccvitem{Morada}{\textit{(não exibida devido a protecção de privacidade)}}
            \ccvitem{Telemóvel}{\textit{(não exibido devido a protecção de privacidade)}}
            \ccvitem{Email}{\textit{(não exibido devido a protecção de privacidade)}}
        }
        \ccvitem{Website}{\weblink{http://ricardoamaral.net}}
        \ccvitem{Nacionalidade}{Portuguesa}
        \ccvitem{Data de Nascimento}{5 de Maio, 1984}
        \ccvitem{Sexo}{Masculino}

        % Occupational Field

        \tabularnewline
        \ccvitem{\large\textbf{Áreas de Competência}}{\large\textbf{Engenharia de Software / Programador}}

        % Work Experience

        \ccvsection{work}{Experiência Profissional}

        \ccvitem{Datas}{Setembro 2011 $\longrightarrow$ Presente}
        \ccvitem{Função ou Cargo Ocupado}{\textbf{Programador Android} \textit{(Conta Própria)}}
        \ccvitem{Principais Actividades e Responsabilidades}{Concepção e desenvolvimento de aplicações para a plataforma Android com os mais altos requisitos de qualidade. Alcançado os mesmos com as melhores práticas desenvolvimento e padrões de desenho. Actualmente com uma única aplicação publicada na Play Store denominada de `Notification Agenda'.}
        \ccvitem{Tipo de Empresa ou Sector}{Desenvolvimento de Aplicações Móveis}

        \tabularnewline

        \ccvitem{Datas}{Outubro 2004 $\longrightarrow$ Janeiro 2008}
        \ccvitem{Função ou Cargo Ocupado}{\textbf{Programador Web}}
        \ccvitem{Principais Actividades e Responsabilidades}{Desenvolvimento de um sistema de gestão de conteúdos (CMS) incluindo diversas aplicações web como a gestão do conteúdo principal, gestão de notícias, gestão de portfólio, formulário dinâmico de candidaturas online e \textit{newsletters}. O sistema foi construído com recurso à linguagem PHP e destinado a uma variedade de websites.}
        \ccvitem{Nome e Endereço do Empregador}{HighBrand (\weblink{http://www.hb.com.pt})\newline 1700-116 Lisboa}
        \ccvitem{Parceiro}{Rui Rocha (\maillink{rui.rocha@hb.com.pt})}
        \ccvitem{Tipo de Empresa ou Sector}{Desenvolvimento de Software}

        % Education and Training

        \ccvsection{education}{Educação e Formação}

        \ccvitem{Datas}{Setembro 2005 $\longrightarrow$ Julho 2012}
        \ccvitem{Designação da Qualificação Atribuída}{\textbf{Licenciatura em Engenharia Informática}}
        \ccvitem{Principais Disciplinas / Competências Profissionais}{Elementos de Engenharia de Sistemas; Programação Funcional; Programação Imperativa; Sistemas de Computação; Algoritmos e Complexidade; Arquitectura de Computadores; Cálculo de Programas; Comunicação de Dados; Programação Orientada aos Objectos; Sistemas Operativos; Bases de Dados; Computação Gráfica; Comunicações por Computador; Desenvolvimento de Sistemas de Software; Processamento de Linguagens; Redes de Computadores; Sistemas de Representação de Conhecimento e Raciocínio; Sistemas Distribuídos.}
        \ccvitem{Nome e Tipo da Organização de Ensino ou Formação}{Universidade do Minho\newline 4710-057 Braga}

        \tabularnewline

        \ccvitem{Datas}{Setembro 1999 $\longrightarrow$ Julho 2005}
        \ccvitem{Designação da Qualificação Atribuída}{\textbf{Graduação no Curso Tecnológico de Informática}}
        \ccvitem{Principais Disciplinas / Competências Profissionais}{Técnicas e Linguagens de Programação; Aplicações Informáticas; Introdução às Tecnologias de Informação; Estrutura e Organização e Tratamento de Dados.}
        \ccvitem{Nome e Tipo da Organização de Ensino ou Formação}{Escola Secundária José Régio\newline 4480-794 Vila do Conde}

        \tabularnewline

        \ccvitem{Data}{Novembro 2003}
        \ccvitem{Evento}{\textbf{Seminário ``Empreendedorismo - Inovação em Movimento''}}
        \ccvitem{Organização da Formação}{Academia dos Empreendedores}

        % Personal Skills and Competences

        \ccvsection{skills}{Aptidões e Competências Pessoais}

        \ecvmothertongue{\normalsize Português}
        \tabularnewline

        \ecvitem{Outras Línguas}{}
        \ecvlanguageheader{(*)}
        \ecvlanguage{Inglês}{\ecvCOne}{\ecvCOne}{\ecvBTwo}{\ecvBTwo}{\ecvCOne}
        \ecvlanguagefooter{(*)}

        \tabularnewline

        \ccvitem{Aptidões e Competências Sociais}{
            \vspace{-12pt}
            \begin{itemize}
                \item Boa capacidade de trabalhar individualmente ou em equipa com um forte sentido de responsabilidade em qualquer das situações;
                \item Capacidade de reagir bem sobre pressão preservando o rigor profissional;
                \item Excelente capacidade de argumentação e comunicação aliados a um espírito crítico;
                \item Capacidade de tomar a iniciativa e efectuar decisões diligentes e cuidadas.
            \end{itemize}
        }

        \ccvitem{Aptidões e Competências de Organização}{
            \vspace{-12pt}
            \begin{itemize}
                \item Capacidade de gestão de projectos com sentido de organização e competência;
                \item Facilidade na transmissão de conhecimentos e metodologias de desenvolvimento;
                \item Elevada capacidade de auto-crítica e auto-avaliação do trabalho efectuado.
            \end{itemize}
        }

        \ccvitem{Aptidões e Competências Informáticas}{
            \vspace{-12pt}
            \begin{itemize}
                \item Linguagens e/ou Tecnologias (Intermédio-Avançado): PHP, C\#, Java, HTML, CSS, \mbox{JavaScript}, jQuery, JSON e SQL;
                \item Linguagens e/ou Tecnologias (Básico-Intermédio): C/C++, VB.NET, Haskell, Prolog, XML e LaTeX;
                \item Linguagens e/ou Tecnologias (Outras): ASP.NET, ActionScript, Pascal, PL/SQL, UML, Razor e Swing;
                \item Sistemas de Gestão de Bases de Dados: MySQL, Oracle, SQL Server and SQLite;
                \item Sistemas de Controlo de Versões: Git e SVN;
                \item Diversos: Android SDK, OpenGL, MVC, Apache HTTP Server, Flex (Gerador de Análise Léxica), Yacc (Gerador de Compilador) e Expressões Regulares.
            \end{itemize}
            \\ &
            \vspace{-12pt}
            \begin{itemize}
                \item Sistemas Operativos: Microsoft Windows, Mac OS X e Linux;
                \item Ambientes de Desenvolvimento Integrado: Microsoft Visual Studio, Eclipse, NetBeans e Oracle SQL Developer;
                \item Software de Produtividade: Microsoft Office, OpenOffice/LibreOffice, Adobe Photoshop, Adobe Illustrator, Adobe Lightroom, Adobe Premiere Pro, Adobe After Effects, Adobe Flash e Adobe Dreamweaver;
                \item Software Computacional: Mathematica e MATLAB.
            \end{itemize}
        }

        \ccvitem{Aptidões e Competências Artísticas}{
            \vspace{-12pt}
            \begin{itemize}
                \item Fotografia.
            \end{itemize}
        }

        \ccvitem{Outras Aptidões e Competências}{
            \vspace{-12pt}
            \begin{itemize}
                \item Facilidade em investigação e aprendizagem de forma autónoma.
            \end{itemize}
        }

        \ccvitem{Carta de Condução}{Categoria B}

        % Additional Information

        \ccvsection{extra}{Informações Adicionais}

        \ccvitem{Publicações}{}
        \ccvitem{}{«Sistema básico de templates em PHP», Revista PROGRAMAR (9ª Edição), Comunidade \href{http://www.portugal-a-programar.pt}{Portugal-a-Programar}, Julho 2007.}

        \ccvitem{Concursos de Programação}{}
        \ccvitem{}{«Concurso de Programação androidPT», Aplicação Notification Agenda, 2º Classificado, Comunidade \href{http://www.androidpt.com}{AndroidPT}, Novembro 2011.}
        \ccvitem{}{«Torneio de Programação para Alunos do Secundário (ToPAS)», 3º Classificado, Departamento de Ciências de Computadores da Faculdade de Ciências da Universidade do Porto, Maio 2005.}
        \ccvitem{}{«Torneio de Programação para Alunos do Secundário (ToPAS)», 4º Classificado, Departamento de Ciências de Computadores da Faculdade de Ciências da Universidade do Porto, Maio 2004.}

        % Appendices

        \ccvsection{appendices}{Anexos}

        \ccvitem{Anexo I}{Projectos Pessoais Desenvolvidos}
        \ccvitem{Anexo II}{Projectos Universitários Desenvolvidos}

        \newpage
        
        % Insert the projects file contents in the document

        % Appendix 1: Personal Projects Developed

\ccvattachment{1}{Anexo I}{\textbf{Projectos Pessoais Desenvolvidos}}

\ccvitem{\textbf{Notification Agenda}\\ Aplicação Android}{\href{http://play.google.com/store/apps/details?id=net.ricardoamaral.apps.notificationagenda}{http://play.google.com/\ldots/details?id=net.ricardoamaral.apps.notificationagenda}}
\ccvitem{Tecnologias Usadas}{Android SDK}
\ccvitem{Sumário do Projecto}{Aplicação simples que transforma pequenas notas em agradáveis notificações na barra de estado com elegantes ícones e de alta qualidade.}

\fancyfoot{}

\tabularnewline

\ccvitem{\textbf{FireNotes}\\ Aplicação Windows}{}
\ccvitem{Tecnologias Usadas}{C\#}
\ccvitem{Sumário do Projecto}{Uma simples ferramenta com um aspecto gráfico personalizado que permite tirar notas (apontamentos) de forma prática e organizada.}

\tabularnewline

\ccvitem{\textbf{Network Switcher}\\ Aplicação Windows}{}
\ccvitem{Tecnologias Usadas}{VB.NET}
\ccvitem{Sumário do Projecto}{Esta aplicação com um interface intuitivo permite a configuração das placas de rede do computador, permitindo a troca de configurações através de perfis.}

\tabularnewline

\ccvitem{\textbf{Easy Settings}\\ Biblioteca .NET}{}
\ccvitem{Tecnologias Usadas}{C\# / XML}
\ccvitem{Sumário do Projecto}{Biblioteca fácil e simples de usar para gerir definições/preferências de qualquer aplicação desenvolvida em linguagens .NET, guardando-as num ficheiro XML.}

\tabularnewline

\ccvitem{\textbf{Outros Projectos}}{Para além dos projectos mais relevantes e destacados acima, existem outros que podem ser consultados no meu perfil do \href{http://github.com/rfgamaral}{GitHub} e/ou no meu site pessoal; conforme estes tenham sido desenvolvidos num âmbito de código aberto ou não.}

% Appendix 2: University Projects Developed

\ccvattachment{2}{Anexo II}{\textbf{Projectos Universitários Desenvolvidos}}

\ccvitem{\textbf{Sistema de Recomendação de Viagens}\\ Aplicação Windows}{Laboratórios de Informática IV\newline Orlando Belo}
\ccvitem{Tecnologias Usadas}{C\# / UML / SQL Server / ASP.NET / Razor / MVC}
\ccvitem{Sumário do Projecto}{Este projecto consistiu na concepção, análise e especificação de requisitos, desenvolvimento, documentação e manutenção de um sistema de \textit{software}. A ideia apresentada para a realização deste projecto no âmbito do `Sistema de Recomendação de Viagens' proposto foi ``voos de baixo custo na Europa''.}

\tabularnewline

\ccvitem{\textbf{UDP.Friendly}\\ Aplicação Windows}{Comunicações por Computador\newline António Costa}
\ccvitem{Tecnologias Usadas}{C\# / UDP}
\ccvitem{Sumário do Projecto}{O objectivo principal deste projecto consistiu no desenvolvimento de uma camada protocolar sobre o UDP que permitisse regular o débito de dados a transmitir em função dos níveis de carga na rede. Como prova de conceito foram desenvolvidas duas aplicações --- segundo o modelo cliente-servidor --- permitindo a transferência de dados.}

\tabularnewline

\ccvitem{\textbf{Treasure: Planet XPTO}\\ Jogo Windows}{Computação Gráfica\newline António Ramires Fernandes}
\ccvitem{Tecnologias Usadas}{C \& C++ / OpenGL \& GLUT}
\ccvitem{Sumário do Projecto}{Este projecto teve como intuito o desenvolvimento de um jogo ao estilo de um \textit{Role-Playing Game} em primeira e terceira pessoa. O objectivo passou por empregar os conhecimentos adquiridos ao longo do semestre, envolvendo conceitos como sistemas de coordenadas, iluminação, texturas, \textit{Display Lists}, \textit{View Frustum Culling}, VBOs, etc\ldots}

\tabularnewline

\ccvitem{\textbf{Turtly Turtle Movie Database}\\ Aplicação Windows e Web}{Bases de Dados\newline José Manuel Machado}
\ccvitem{Tecnologias Usadas}{C\# / PHP / HTML / CSS / JavaScript \& jQuery / Oracle \& PL/SQL}
\ccvitem{Sumário do Projecto}{Este projecto teve como propósito a aquisição de conhecimentos do SGBD Oracle, tendo como tema a indústria do cinema, mais precisamente os filmes. Foram desenvolvidas duas aplicações, uma para a Web como \textit{front end} para consulta da informação e uma Desktop para administração e manutenção da base de dados.}

\tabularnewline

\ccvitem{\textbf{Algoritmo Simplex}\\ Aplicação Web}{Modelos Determinísticos de Investigação Operacional\newline José António Oliveira}
\ccvitem{Tecnologias Usadas}{PHP / HTML / CSS / JavaScript \& jQuery}
\ccvitem{Sumário do Projecto}{Este projecto consistiu em desenvolver uma aplicação capaz de resolver problemas de programação linear de maximização e minimização. Estes problemas foram resolvidos com recurso a uma implementação do algoritmo Simplex Primal ou Dual.}

\tabularnewline

\ccvitem{\textbf{Só Amigos / Sempre Ligados}\\ Aplicação Linux e Windows}{Laboratórios de Informática III\newline António Nestor Ribeiro \& Luís Paulo Santos}
\ccvitem{Tecnologias Usadas}{C / Java / Swing / MVC}
\ccvitem{Sumário do Projecto}{Associados à mesma Unidade Curricular, foram desenvolvidos dois projectos bastante semelhantes. Ambos consistiram no desenvolvimento ou utilização de estruturas de dados --- recorrendo a diferentes linguagens --- tendo os mesmos como base a implementação de uma versão simplificada da rede social LinkedIn.}

\tabularnewline

\ccvitem{\textbf{AEROGEST}\\ Aplicação Windows}{Programação Orientada aos Objectos\newline F. Mário Martins}
\ccvitem{Tecnologias Usadas}{Java}
\ccvitem{Sumário do Projecto}{Este projecto teve como objectivo aplicar todos os conhecimentos adquiridos ao longo do semestre sobre o paradigma de sistemas de \textit{software} orientado a objectos. O projecto consistiu no desenvolvimento de um ``sistema de gestão de voos''.}

\tabularnewline

\ccvitem{\textbf{Observação}}{Os projectos acima não representam todos os projectos desenvolvidos durante a licenciatura, sendo apenas listados aqueles considerados mais importantes ou relevantes.}

    \end{europecv}

\end{document}